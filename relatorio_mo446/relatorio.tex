\documentclass[a4paper]{article}

\usepackage[english]{babel}
\usepackage[utf8]{inputenc}
\usepackage{amsmath}
\usepackage{graphicx}
\usepackage{cite}

\title{Hand Tracking Using Distinct OpenCV Methods and Analysis of their Accuracy and Performance}

\author{Francisco Jose Nardi Filho\\ Lucas Gasparetto Farris}

\date{June  24, 2015}

\begin{document}
\maketitle

\begin{abstract}
\noindent
Mean Shift, Adaptative Mean Shift, Camshift using SIFT keys and Particle Filtering are four successful approaches to visual tracking. 
However, each of them have their particular strengths and weaknesses. In this paper, we intend to present implementations of each distinct method from the OpenCV library and make an accuracy and  performance analysis for a hand tracking case. Fifteen videos taken from YouTube showing diverse situations as teaching of the American signal language alphabet, Mudras (gestures from an Indian dance), as well as guitar playing  were used as dataset to run the algorithms and perform the hand tracking. The hand position on these one-minute videos was manually annotated prior to the methods' accuracy and performance analysis. Extensive experimental results demonstrate that XXX outperforms YYY in hand tracking considering the provided database. Our approach produces reliable tracking while effectively handling rapid motion.
\end{abstract}

\section{Introduction}

Object tracking is to determine the position of the object at each frame of a video sequence. Reliable visual tracking is indispensable in several fields such as: ambient intelligent systems, augmented reality, human–computer interaction, video compression, and robotics. However, the task of robust tracking is very challenging regarding fast motion, occlusion, structural deformation, illumination variation, background clutters, and real-time restriction \cite{Yin2011}. 
\par
Tracking algorithms can be divided into two categories. The first category is deterministic method. Mean-Shift  is a typical example. This method finds the local maximum of probability distribution in the direction of gradient. It is usually more accurate and quick in tracking than the probabilistic multi-hypothesis tracking algorithm. But it may run into trouble when similar objects are presented in background or when complete occlusion occurs. Camshift is called the Continuously Adaptive Mean Shift algorithm which is a modified algorithm of Mean-Shift. The second category is probabilistic method. The representative method is particle filter which is a multi-hypothesis tracking algorithm under the Bayesian framework . Due to particle filter’s non-Gaussian non-linear assumption and multiple hypothesis property, they have been successfully applied to visual tracking, and show unique merit in cluttered environment. However, the inefficiency in sampling (due to the problem of degeneracy and impoverishment) and the huge computational complexity limit the usefulness of particle filter in on-line tracking \cite{Yin2011}.
\par

\section{Related work}
\section{Methods}
\section{Experiments}
\section{Results and discussion}
\section{Conclusion}

\bibliographystyle{plain}
\bibliography{mylib}

\citation{Wang2009}
\citation{Kolsch2004}
\citation{Min1997}
\citation{Maggio2005}
\citation{Yin2011}
\citation{Shan2004}
\citation{Shan2007}
\citation{Ning2012}

\end{document}